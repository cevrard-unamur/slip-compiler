\section{Architecture du compilateur}

Notre Compilateur est divisé en 6 p\emph{ackages}:
\begin{itemize}
    \item Application : qui contient toutes les classes liées à la sauvegarde des symboles
    \item Exception : qui contient l'exception \texttt{PlayPlusException}
    \item Language : qui contient le visiteur du language et tous les \emph{helpers} qui aident le visiteur
    \item Main : qui contient la fonction main, le parseur \emph{ANTLR} et le \emph{listener} d'erreurs
    \item Map : qui contient le \emph{Listener} de la carte
    \item NBC : qui contient le \emph{Printer} NBC, le \emph{Visiteur} NBC, les \emph{Helper} et \emph{Writer} qui vont aider les deux premières classes
\end{itemize}

\subsection{Application}

Voici les différentes \emph{class} dans le \emph{Package} Application.

\subsubsection{Application}

Cette \emph{class} permet d'initialiser une nouvelle instance d'\texttt{Application}, elle contient aussi les méthodes permettant d'entrer dans le bon contexte et d'en sortir.
Elle contient aussi les méthodes d'ajout des différents symboles et celles qui permettent d'y accéder.

\subsubsection{Array}

Cette \emph{class} permet d'initialiser un \texttt{Array} qui sera utilisé dans un contexte.

\subsubsection{Context}

\emph{Class} principale qui initialise un \texttt{Context} qui va être utilisé dans une \texttt{Application}. Ce \texttt{Context} contient nos 4 \texttt{Hashtable} dans lesquels nous sauvegardons nos symboles.

Ces 4 \texttt{Hashtables} sont les suivantes:

\begin{itemize}
    \item variableSymbols : qui va contenir tous les symboles des variables.
    \item functionSymbols : qui va contenir tous les symboles des fonctions.
    \item recordSymbols : qui va contenir tous les symboles des records.
    \item enumSymbols : qui va contenir tous les symboles des énumérations.
\end{itemize}

Elles sont épaulées par 4 \texttt{Array}:
\begin{itemize}

    \item variables : qui contient les différentes variables du programme.
    \item functions : qui contient les différentes fonctions du programme.
    \item records : qui contient les différentes structures du programme.
    \item enums : qui contient les différentes énumérations du programme.
\end{itemize}

De plus, la carte est aussi ajoutée dans cette \emph{class}.

Finalement, en plus de cela, la \emph{class} contient les méthodes qui permettent d'ajouter et retourner les \texttt{Variable},\texttt{Array},\texttt{Function} et \texttt{Enum}.

\subsubsection{Enumeration}

Cette classe permet d'initialiser une \emph{Enumeration}.

\subsubsection{Function}

Cette classe permet d'initialiser une \emph{Function}. Elle \emph{extends} \texttt{Context} car elle doit pouvoir stocker les mêmes symboles qu'un contexte. En plus du contexte, elle doit stocker ses arguments et son type de retour.
Toutes les méthodes liées à cette classe servent à ajouter/retourner les divers symboles spécifiques à la fonction.

\subsubsection{Record}

Cette classe permet d'initialiser un \texttt{Record}.

\subsubsection{VariableBase}

Cette classe permet d'initialiser une \texttt{VariableBase}. Qui est le type de base d'une variable.

\subsubsection{Variable}

Cette classe \emph{extends} \texttt{VariableBase} et ajoute juste un attribut \texttt{isConstant} qui est là pour savoir si la variable est une constante ou non.

\subsection{Exception}

Ce package ne contient que \texttt{PlayPlusException} qui \emph{extends RuntimeException} et qui est lancé quand il y a une exception non attendue dans le \emph{Parser}.

Nous utilisons une \texttt{RuntimeException} afin de ne pas avoir à l'ajouter dans les signatures des méthodes surchargées.

\subsection{Language}

Ce package contient le \texttt{LanguageVisitor} qui va visiter l'arbre syntaxique et en extraire les symboles utiles. Le package contient un sous-package \emph{Helper} qui contient toutes les méthodes qui vont
être utilisées pour récupérer les symboles.

\paragraph{ActionExpression}

Cette classe contient les méthodes qui servent à \emph{parser} les \texttt{ActionExpression}.

\paragraph{ArrayHelper}

Cette classe contient la méthode qui sert à convertir x noeuds terminaux \texttt{NUMBER} en un tableau d'entiers.

\paragraph{BooleanExpression}

Cette classe contient les méthodes qui servent à \emph{parser} n'importe quelle expression booléenne.

\paragraph{CharExpression}

Cette classe contient les méthodes qui servent à \emph{parser} n'importe quelle expression de caractère.

\paragraph{ConstantExpression}

Cette classe contient les méthodes qui servent à \emph{parser} n'importe quelle expression de constante.

\paragraph{DeclarationExpression}

Cette classe contient les méthodes qui servent à \emph{parser} n'importe quelles Déclaration et Instruction.

\paragraph{EnumExpression}

Cette classe contient les méthodes qui servent à \emph{parser} n'importe quelle énumération.

\paragraph{ForExpression}

Cette classe contient les méthodes qui servent à \emph{parser} n'importe quelle expression \texttt{for}.

\paragraph{FunctionExpression}

Cette classe contient les méthodes qui servent à \emph{parser} n'importe quelle expression de fonction, argument, appel de fonction ou déclaration de fonction.

\paragraph{GenericExpression}

Cette classe contient les méthodes qui servent à \emph{parser} n'importe quelle expression gauche, de comparaison ou de parenthèses.

\paragraph{IfExpression}

Cette classe contient les méthodes qui servent à \emph{parser} n'importe quelle instruction \texttt{if}.

\paragraph{IntegerExpression}

Cette classe contient les méthodes qui servent à \emph{parser} n'importe quelle expression d'\texttt{integer}.

\paragraph{RepeatExpression}

Cette classe contient les méthodes qui servent à \emph{parser} n'importe quelle expression \texttt{repeat}.

\paragraph{StructureExpression}

Cette classe contient les méthodes qui servent à \emph{parser} n'importe quelle structure.

\paragraph{Tuple}

Cette classe permet de créer un \texttt{Tuple} de deux items de type différent.

\paragraph{VariableExpression}

Cette classe contient les méthodes qui servent à \emph{parser} les instructions, les définitions, les initialisations de variables et de tableaux.

\paragraph{VariableHelper}

Cette classe contient les méthodes qui servent à ajouter une variable, un tableau et une structure. Elles sont utilisées dans \texttt{VariableExpression} et \texttt{ConstantExpression}.

\paragraph{WhileExpression}

Cette classe contient les méthodes qui servent à \emph{parser} n'importe quelle expression \texttt{while}.

\subsection{Main}

Ce package contient la fonction \texttt{Main} le \emph{paser ANTLR} et le \emph{listener} d'erreurs.

\subsection{Map}

Ce package ne contient qu'une classe qui est le \texttt{MapListener} qui va \emph{parser} la carte , sauvegarder ses symboles dans un tableau et vérifier qu'elle est correcte.

\subsection{NBC}

Ce package contient deux classes principales \texttt{NBCVisitor} et \texttt{NBCPrinter} qui vont respectivement visiter notre programme et le convertir en instructions \emph{NBC}. Les classes aidant sont réparties en deux packages.

\subsubsection{Helper}

Premier package qui contient toutes les classes d'aide.

\paragraph{ActionExpression}

Cette classe contient les méthodes qui servent à \emph{parser} n'importe quelle action en code \emph{NBC}.

\paragraph{ActionInterface}

Classe qui permet d'exécuter le \emph{Writer}.

\paragraph{ArrayExpression}

Cette classe contient les méthodes qui servent à \emph{parser} n'importe quelle \emph{Array} en code \emph{NBC}.

\paragraph{AssignationExpression}

Cette classe contient les méthodes qui servent à \emph{parser} n'importe quelle assignation en code \emph{NBC}.

\paragraph{ComparisonExpression}

Cette classe contient les méthodes qui servent à \emph{parser} n'importe quelle comparaison en code \emph{NBC}.

\paragraph{DeclarationExpression}

Cette classe contient les méthodes qui servent à \emph{parser} n'importe quelle déclaration en code \emph{NBC}.

\paragraph{ForExpression}

Cette classe contient les méthodes qui servent à \emph{parser} n'importe quel \texttt{for} en code \emph{NBC}.

\paragraph{FunctionExpression}

Cette classe contient les méthodes qui servent à \emph{parser} n'importe quelle fonction en code \emph{NBC}.

\paragraph{IfExpression}

Cette classe contient les méthodes qui servent à \emph{parser} n'importe quel \texttt{if} en code \emph{NBC}.

\paragraph{IntegerExpression}

Cette classe contient les méthodes qui servent à \emph{parser} n'importe quelle expression entière en code \emph{NBC}.

\paragraph{IntegerHelper}

Cette classe sert à renvoyer le bon symbole lors d'une opération entière.

\paragraph{LeftExpression}

Cette classe contient les méthodes qui servent à \emph{parser} n'importe quelle expression gauche en code \emph{NBC}.

\paragraph{MapExpression}

Cette classe contient les méthodes qui servent à importer la carte.

\paragraph{RepeatExpression}

Cette classe contient les méthodes qui servent à \emph{parser} n'importe quel \texttt{repeat} en code \emph{NBC}.

\paragraph{RightExpression}

Cette classe contient les méthodes qui servent à \emph{parser} n'importe quelle expression droite en code \emph{NBC}.

\paragraph{VariableExpression}

Cette classe contient les méthodes qui servent à \emph{parser} n'importe quelle variable en code \emph{NBC}.

\paragraph{VariableHelper}

Cette classe sert à renvoyer le bon symbole dépendant le type de variable.

\paragraph{WhileExpression}

Cette classe contient les méthodes qui servent à \emph{parser} n'importe quelle expression \emph{While} en code \emph{NBC}.

\subsubsection{Writer}

Ce package contient toutes les classes servant à écrire le code \emph{NBC}.

\paragraph{ActionWriter}

Cette classe permet d'écrire n'importe qu'elle type d'action.

\paragraph{FunctionWriter}

Cette classe permet d'écrire n'importe qu'elle type de fonction.

\paragraph{IfWriter}

Cette classe permet d'écrire n'importe qu'elle type de \texttt{if}.

\paragraph{LogicWriter}

Cette classe permet d'écrire n'importe qu'elle type de comparaison logique.

\paragraph{LoopWriter}

Cette classe permet d'écrire n'importe qu'elle type de boucle.

\paragraph{NBCCodeTypes}

Cette classe renvoie la bonne écriture du type de variable en code \emph{NBC}.

\paragraph{NBCIntCodeTypes}

Cette classe renvoie la bonne écriture du type d'opération entière en code \emph{NBC}.

\paragraph{NBCOpCodeTypes}

Cette classe renvoie la bonne écriture du type d'opérateur logique en code \emph{NBC}.

\paragraph{NBCWriter}

Cette classe écrit le code \emph{NBC} de base inhérent au fonctionnement du robot.

\paragraph{OperationWriter}

Cette classe permet d'écrire n'importe qu'elle type d'opération en code \emph{NBC}.

\paragraph{PreprocessorWriter}

Cette classe permet d'écrire les informations pour le préprocesseur en code \emph{NBC}.

\paragraph{VariableWriter}

Cette classe permet d'écrire n'importe qu'elle type de variable en code \emph{NBC}.