\section{Structure de la table des symboles}

% Une description de la structure de donn ́ee et de l’utilisation de votre table des symboles avec la/les d ́efinition(s) du/des classe(s) utilis ́ee(s) pour l’impl ́ementer et 
% une justifica- tion de votre choix. Inutile de reprendre ici l’ensemble de la documentation, le but est d’expliquer comment se structure la table des symboles et comment
% elle a  ́eté utilisée pour valider le code en entr ́ee du compilateur (pensez donc `a utiliser un formalisme ad ́equat, tel qu’un diagramme de classes).

La structure ustilisée se trouve dans la classe \emph{Context.java}.

La "Table des symboles" en elle même est composée de quatre \emph{Hashtables} qui sont accompagnées d'un \emph{array} pour chanque symbole.
Les quatre symboles différents sont :
\begin{itemize}
    \item VariableBase
    \item Function
    \item Record
    \item Enumeration
\end{itemize}

Notre choix d'utiliser quatre \emph{Hashtable} est simple, notre point de vue est qu'il y a quatre classe de symboles. Ansi donc pour avoir une simplicité de codage, de stockage et
d'acces aux symboles, nous avons utilisé cette implémentation.

Notre "Table des symbole" est donc utilisée comme suit:
\begin{itemize}
    \item Lors du passage de notre visiteur, chaque symbole est stoqué dans l'\emph{array} assigné à son symbole et référencé dans sa \emph{Hashtable}. Le tout dans le contexte dans lequel se trouve ce symbole.
    \item Lors du "test" du code, chaque \emph{Function} est vérifiée par la validité de son type de retour et de la bonne présence des \emph{VariableBase} dans leur contexte respectif pour qu'elles respectent leurs portées. Cette action est facilitée par la présence des \emph{Hashtable}
    \item Chaque symbole est donc restitué en code NBC en conservant chaque fonctionalité du code \emph{Slip} entré dans le compilateur.
\end{itemize}

Ceci est donc le résumé de l'implémentation de notra "Table des symboles".