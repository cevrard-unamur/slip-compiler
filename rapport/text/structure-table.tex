\section{Structure de la table des symboles}

La structure utilisée se trouve dans la \emph{class} \texttt{Context.java}.

La "Table des symboles" en elle même est composée de quatre \texttt{Hashtables} qui sont accompagnées d'un \emph{array} pour chaque symbole.

Les quatre symboles différents sont :
\begin{itemize}
    \item VariableBase
    \item Function
    \item Record
    \item Enumeration
\end{itemize}

Notre choix d'utiliser quatre \texttt{Hashtable} est simple, notre point de vue est qu'il y a quatre \emph{class} de symboles. Ainsi donc pour avoir une simplicité de codage, de stockage et d'accès aux symboles, nous avons utilisé cette implémentation.
Cela permet aussi de facilement éviter les variables ou function avec des noms similaires.

Notre "Table des symboles" est donc utilisée comme suit:
\begin{itemize}
    \item Lors du passage de notre visiteur, chaque symbole est stocké dans l'\texttt{Array} assigné à son symbole et référencé dans sa \texttt{Hashtable}. Le tout dans le contexte dans lequel se trouve ce symbole.
    \item Lors de la vérification de la sémantique du code, chaque \texttt{Function} est vérifiée par la validité de son type de retour et de la bonne présence des \texttt{VariableBase} dans leur contexte respectif pour qu'elles respectent leurs portées. Cette action est facilitée par la présence des \texttt{Hashtable.}
    \item Chaque symbole est donc restitué en code NBC en conservant chaque fonctionnalité du code \textbf{Slip} entré dans le compilateur.
\end{itemize}

Ceci est donc le résumé de l'implémentation de notre "Table des symboles".