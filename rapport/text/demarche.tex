\section{Démarche générale}

% Une description de la d ́emarche g ́en ́erale que vous avez adopt ́ee pour construire votre compilateur, ainsi que les choix faits pour l’impl ́ementation. On vous demande ici de jeter un regard critique sur la mani`ere dont le projet s’est d ́eroul ́e au sein de votre groupe : quelle d ́emarche avez-vous adopt ́ee? comment vous ˆetes-vous organis ́es pratiquement?  ́etait-ce, `a posteriori, une bonne id ́ee ? etc.

Notre groupe était composé de deux personnes. La démarche que nous avons adopté a été de laisser Cédric commencer à travailler en chaque début de partie afin de créer l'architecture de l'application ainsi que la structure des différents packages. Une fois que cette structure était créée, une répartition était faite du travail.
 
 
 Il a parfois été difficile de découper la réalisation du travail de manière équitable. En effet, une dépendance assez importante dans la grammaire de la partie \emph{expression droite} à fait que celle-ci devait souvent être réaliser avant tout le reste et conditionnait énormément la structure du projet.
 
 Avant la fin de la phase 2, un gros travail de refactoring a été effectué afin de rendre le code plus lisible et d'en faciliter la maintenance.